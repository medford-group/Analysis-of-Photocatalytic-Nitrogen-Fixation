%%%%
%%%%
%%%%
Easy:

4. The band gap shown in Figure 3 is much larger than 3.0 eV, which is usually reported optical band gap of rutile TiO2.

2. I am curious how the authors have hypothesized that rutile (110) is active surface for nitrogen fixation. Please describe the reason in details. %

3. Band edges of rutile TiO2 and redox potentials are shown in Figure 3, but there is no citation. Necessary citations are suggested to be added in the manuscript. %

7. Equations 7–12 are incorrect. (we put the dissociative mechanism twice) %

11. There are typo, for example, "the the" in Line 2 on page 6. %

12. Abbreviation should be explained when it first appears in the text.

a.      It is unclear what the difference between the “gas” and “aqueous” conditions are in the discussion of these results.  The authors note that the gas phase is humidified, and presuming 100 percent humidity, the chemical potential of gas phase water and liquid water would be equivalent. The chemical potential of water in the aqueous system is noted as being calculated directly from the saturation pressure of water (ie, 100 percent RH), and from what I can tell, the water chemical potential is the only thing that could differentiate the two conditions in their modeling approach.   Did the authors consider the “gas” conditions as 0 percent relative humidity, and the “aqueous” conditions as 100 percent relative humidity?  If so, this should be clarified, and discussed why 0 percent RH would be the relevant “gas” phase conditions given this system would presumably be exposed to ambient air. (we need to be more clear in the caption, the N2 plot is at 100 percent RH and varies N2, the other is a atmospheric gas with water being added successively) %

b.      In Figure 2, to make the plots on the left (a, b, and c), I believe a constant value of the chemical potential of N2 must have been assumed.  On the right (d-f), a constant chemical potential of water must have been assumed.  The authors should clarify what values were used for these species in making these plots. (yes, this is the case) %


5)       Bottom of page 22, “reference hydrogen electrode” should be “reversible hydrogen electrode.” %
6)      Page 23 – The potentials given for holes and excited electrons should be clarified as to what is the reference scale. %

%%%%
%%%%
%%%%
Medium:

1. Abstract should be improved to emphasis the importance of the study.
5. It is well known that pristine rutile is not active for hydrogen evolution reaction. Therefore, selective reduction of nitrogen to ammonia over pristine TiO2 may not be challenging even though the redox potential is near SHE. (I don't think that's well known, and I'm not sure what to do with this comment.)

6. The first step for dissociative nitrogen reduction is scission of the N-N bond. This step is not photoexcited process of TiO2. Therefore, the reviewer thinks that the high energy barrier is routinely accepted. Are there any pathway for the activation of nitrogen by photoexcited electrons? For example, N2* + * + H+ + e-  NH* + N*.(not sure what he wants, we could make this figure but it's plain to see from figure 4 that it wouldn't matter, it's still very uphill)

10. It is suggested that nitrogen is first oxidized to nitric oxide and subsequently reduced to ammonia. If so, there are reduction pathways of NO into N2 and N2O in addition to NH3. Detailed discussions are necessary since the proposed mechanism is speculative. (this is easy, all of our paths start at N2, NO->N2 is the reverse of one of our path. NO->N2O->N2 has been observed by Yates experimentally, it is not speculative)

9. Figure 10 was not referred in the manuscript.

c.      In making these plots, the authors have only considered adsorbed N2, whereas the steady state system operating photocatalytically would attain a surface potential, and formation of stable NxHy species would be possible.  As the authors show, adsorbed NH3 (and other species under certain conditions) can be more stable than adsorbed N2 under operating conditions.  The authors might consider making diagrams at a presumed operating potential to consider if the surface would attain (thermodynamically) a high coverage of NHx species.  Similarly, operating conditions could cause the reduction of surface O atoms (ie, the adsorption of H or formation of surface O vacancies), and this was not considered. (I think we should write something like "because only thermodynamics is presented and the process is known to be slow, we assume that the process of nitrogen fixation has a significant barrier. Thus we will limit our surface analysis to reactants" if we show NxHy species it would show those only and likely be an inaccurate mess)


3)      The authors state in the conclusions “defects are not predicted to be thermodynamically stable under operating conditions.”  The basis for this conclusion is unclear – I don’t see a consideration of the reduction of surface O atoms to form water and a vacancy considered in the paper. (we did this O->H2O analysis implicitly when we calculated the energy to form the defect. Now that I read this a little more, we might have reason to believe these defects do exist on the surface in a real system. But this comment was beside the point, because the pathway wasn't possible even with it being present.)

%%%%
%%%%
%%%%
Hard:

13. The format of the references should be unified according to the requirements of this journal. (I'll have to check through all the references)

2)       I am a bit confused as to the distinction between routes that first undergo oxidation and those that initiate through reduction, given the surface of TiO2 has O atoms.  It appears the difference between N2O* and N2* is that an excess O atom is present.  However, presumably N2* could form by creating an O vacancy, appearing as “N2O* adsorbed with a vacancy nearby.”  Or, wouldn’t N2 reduction occurring on a surface that had excess O* to begin with be mechanistically equivalent to the “NO” route?  Presumably N2 would adsorb to an O rather than the Ti if the surface was overoxidized. (in the first part of his comment he is thinking of a mars van krevelen mechanism, we didn't consider this because we didn't need to. I'm confused about what he's saying in the second part, but what I'm reading is instead of having N2* and O* reacting on the surface we just have a path to O*, then that acts as an adsorption site. He's right that this would be lower in energy. Not sure what to do with this knowledge though.)



%%%%
%%%%
%%%%
Unsure:

8. In the caption of Figure 9, "the conduction band edge" may be "the valence band edge".(okay, not sure what action we are meant to take)

I understand that the authors have chosen to remove the experimental data which was superfluous due to already published data by others as pointed out by one of the reviewers. As it stands, I do not recommend this paper for publication in EES since it only presents computational data that does not explain the experimental findings nor suggests a reaction mechanism instead it focuses on excluding a number of reaction mechanisms. Therefore, I think this manuscript would be better suited in a more technical computational journal.

4)      The consideration of Fe doping is rather basic, as the authors do not appear to have considered whether Fe doping would alter the stable state of the surface.  Fe doping forces formation of a Fe4+ formal species, though reduction to Fe3+ or Fe2+ would be much more likely under N2 reduction conditions (see Fe Pourbaix diagram).  A Fe2+ model would be relatively easy for the authors to consider, as this would likely form simply by having Fe doping occur together with an O vacancy formation.   Fe3+ could be modeled with two Fe dopants and a vacancy.  These would be much more realistic models of the Fe doped system.  Some discussion of this large limitation in the Fe doping consideration should be added, or possibly the Fe doping analysis simply removed.  In its current form, I would consider the Fe doping calculations done possibly misleading. (I don't know about iron doping chemistry to address this comment.)

a.      In the Fe doped section, I did not understand the statement “The energy required to form this defect was calculated to be 2 eV relative to bulk rutile and BCC iron.”  Such a consideration would require at least knowing the chemical potential of O (or that of water, protons, and an electrochemical potential), so it is unclear the basis on which it was concluded Fe doping was unstable. (we have a zero, we just never stated it. We implicitly assumed there is no change in oxidation state between BCC iron and this state, this may be a bad assumption.)


























%% copy of the original comments below
Reviewer(s)' Comments to Author:

Reviewer: 1

Comments:
This manuscript reports an analysis of the free energy diagrams for photocatalytic conversion of dinitrogen to ammonia over rutile (110) model surface. The results indicate that nitrogen reduction is improbable, but the N-N bond cleavage is thermodynamically facile on rutile (110) through an oxidative pathway with the strong oxidative driving force provided by photogenerated holes. This manuscript may be published in ACS Sustainable Chemistry & Engineering after revisions after addressing the following issues.

1. Abstract should be improved to emphasis the importance of the study.
2. I am curious how the authors have hypothesized that rutile (110) is active surface for nitrogen fixation. Please describe the reason in details.
3. Band edges of rutile TiO2 and redox potentials are shown in Figure 3, but there is no citation. Necessary citations are suggested to be added in the manuscript.
4. The band gap shown in Figure 3 is much larger than 3.0 eV, which is usually reported optical band gap of rutile TiO2.
5. It is well known that pristine rutile is not active for hydrogen evolution reaction. Therefore, selective reduction of nitrogen to ammonia over pristine TiO2 may not be challenging even though the redox potential is near SHE.
6. The first step for dissociative nitrogen reduction is scission of the N-N bond. This step is not photoexcited process of TiO2. Therefore, the reviewer thinks that the high energy barrier is routinely accepted. Are there any pathway for the activation of nitrogen by photoexcited electrons? For example, N2* + * + H+ + e-  NH* + N*.
7. Equations 7–12 are incorrect.
8. In the caption of Figure 9, "the conduction band edge" may be "the valence band edge".
9. Figure 10 was not referred in the manuscript.
10. It is suggested that nitrogen is first oxidized to nitric oxide and subsequently reduced to ammonia. If so, there are reduction pathways of NO into N2 and N2O in addition to NH3. Detailed discussions are necessary since the proposed mechanism is speculative.
11. There are typo, for example, "the the" in Line 2 on page 6.
12. Abbreviation should be explained when it first appears in the text.
13. The format of the references should be unified according to the requirements of this journal.



Reviewer: 2

Comments:
I understand that the authors have chosen to remove the experimental data which was superfluous due to already published data by others as pointed out by one of the reviewers. As it stands, I do not recommend this paper for publication in EES since it only presents computational data that does not explain the experimental findings nor suggests a reaction mechanism instead it focuses on excluding a number of reaction mechanisms. Therefore, I think this manuscript would be better suited in a more technical computational journal.


Reviewer: 3

Comments:
This study applies density functional theory calculations to examine the elementary reaction free energies associated with nitrogen reduction on the rutile TiO2 (110) surface.  The results presented illustrate well that the reduction of nitrogen to ammonia is not energetically feasible on the pristine, stoichiometric surface.  Some consideration of oxygen vacancies and doped Fe looks promising and is considered at a relatively basic level, as well as potential other conversion routes that include first oxidation followed by reduction.  The overall conclusions are interesting, and the resulted are presented with limitations clearly highlighted.  The limitations in the analysis are quite severe, as there is no direct consideration of elementary kinetics (ie, barriers), and catalytic kinetics are, of course, generally dictated by barriers and not reaction energies.  However, in this study, large differences in reaction energies are sufficient to make some interesting qualitative conclusions.  Other major limitations in the modeling approach include the lack of consideration of interfacial solvation and charging in the photoelectrocatalytic system.  Again, though these are significant limitations, the large energetic differences make qualitative conclusions still generally well supported by the more simplistic analysis. As these limitations are highlighted for the reader and discussed, I recommend publication following consideration of the following significant comments.
1)       I have a number of comments regarding the consideration of the stable state of the surface under reaction conditions, the first section of results.
a.      It is unclear what the difference between the “gas” and “aqueous” conditions are in the discussion of these results.  The authors note that the gas phase is humidified, and presuming 100% humidity, the chemical potential of gas phase water and liquid water would be equivalent. The chemical potential of water in the aqueous system is noted as being calculated directly from the saturation pressure of water (ie, 100% RH), and from what I can tell, the water chemical potential is the only thing that could differentiate the two conditions in their modeling approach.   Did the authors consider the “gas” conditions as 0% relative humidity, and the “aqueous” conditions as 100% relative humidity?  If so, this should be clarified, and discussed why 0% RH would be the relevant “gas” phase conditions given this system would presumably be exposed to ambient air.
b.      In Figure 2, to make the plots on the left (a, b, and c), I believe a constant value of the chemical potential of N2 must have been assumed.  On the right (d-f), a constant chemical potential of water must have been assumed.  The authors should clarify what values were used for these species in making these plots.
c.      In making these plots, the authors have only considered adsorbed N2, whereas the steady state system operating photocatalytically would attain a surface potential, and formation of stable NxHy species would be possible.  As the authors show, adsorbed NH3 (and other species under certain conditions) can be more stable than adsorbed N2 under operating conditions.  The authors might consider making diagrams at a presumed operating potential to consider if the surface would attain (thermodynamically) a high coverage of NHx species.  Similarly, operating conditions could cause the reduction of surface O atoms (ie, the adsorption of H or formation of surface O vacancies), and this was not considered.
2)       I am a bit confused as to the distinction between routes that first undergo oxidation and those that initiate through reduction, given the surface of TiO2 has O atoms.  It appears the difference between N2O* and N2* is that an excess O atom is present.  However, presumably N2* could form by creating an O vacancy, appearing as “N2O* adsorbed with a vacancy nearby.”  Or, wouldn’t N2 reduction occurring on a surface that had excess O* to begin with be mechanistically equivalent to the “NO” route?  Presumably N2 would adsorb to an O rather than the Ti if the surface was overoxidized.
3)      The authors state in the conclusions “defects are not predicted to be thermodynamically stable under operating conditions.”  The basis for this conclusion is unclear – I don’t see a consideration of the reduction of surface O atoms to form water and a vacancy considered in the paper.
4)      The consideration of Fe doping is rather basic, as the authors do not appear to have considered whether Fe doping would alter the stable state of the surface.  Fe doping forces formation of a Fe4+ formal species, though reduction to Fe3+ or Fe2+ would be much more likely under N2 reduction conditions (see Fe Pourbaix diagram).  A Fe2+ model would be relatively easy for the authors to consider, as this would likely form simply by having Fe doping occur together with an O vacancy formation.   Fe3+ could be modeled with two Fe dopants and a vacancy.  These would be much more realistic models of the Fe doped system.  Some discussion of this large limitation in the Fe doping consideration should be added, or possibly the Fe doping analysis simply removed.  In its current form, I would consider the Fe doping calculations done possibly misleading.
a.      In the Fe doped section, I did not understand the statement “The energy required to form this defect was calculated to be 2 eV relative to bulk rutile and BCC iron.”  Such a consideration would require at least knowing the chemical potential of O (or that of water, protons, and an electrochemical potential), so it is unclear the basis on which it was concluded Fe doping was unstable.
5)       Bottom of page 22, “reference hydrogen electrode” should be “reversible hydrogen electrode.”
6)      Page 23 – The potentials given for holes and excited electrons should be clarified as to what is the reference scale.
