Reviewer: 1

Comments:
This manuscript reports an analysis of the free energy diagrams for photocatalytic conversion of dinitrogen to ammonia over rutile (110) model surface. The results indicate that nitrogen reduction is improbable, but the N-N bond cleavage is thermodynamically facile on rutile (110) through an oxidative pathway with the strong oxidative driving force provided by photogenerated holes. This manuscript may be published in ACS Sustainable Chemistry & Engineering after revisions after addressing the following issues.

%We appreciate the positive perspective, and the detailed and thoughtful comments below. We have addressed each comment point-by-point and now feel that the manuscript is significantly stronger.

1. Abstract should be improved to emphasis the importance of the study.

% Thank you for this suggestion. We have added two phrases to the abstract to highlight the novelty and importance of this study: "This is the first application of computational techniques to photocatalytic nitrogen fixation" and "This work provides strong evidence against the most commonly reported experimental hypotheses". We hope that this helps potential readers recognize the significance of the work.

2. I am curious how the authors have hypothesized that rutile (110) is active surface for nitrogen fixation. Please describe the reason in details.

%The rutile (110) is the lowest-energy surface of rutile, and has been previously hypothesized by other groups. We have added a discussion of this in the introduction:

%"The rutile (110) surface is hypothesized to be the active surface for due to the fact that photocatalytic nitrogen fixation rates have been observed to correlate with the amount of rutile in TiO$_2$ samples; the (110) surface is lowest energy surface on rutile and is likely to provide a model for other rutile surfaces. Furthermore, the rutile (110) surface has been explicitly hypothesized to be the active surface in recent experimental work."

3. Band edges of rutile TiO2 and redox potentials are shown in Figure 3, but there is no citation. Necessary citations are suggested to be added in the manuscript.

% The reviewer is correct that these citations should appear in the caption, and they have now been added. We note that they previously appeared in the methods section, but realize that this is inconvenient.

4. The band gap shown in Figure 3 is much larger than 3.0 eV, which is usually reported optical band gap of rutile TiO2.

%We thank the reviewer for their astuteness in recognizing this issue. It seems that the band gap was 3.2 eV, consistent with anatase (rather than rutile). This has been fixed in Fig. 3, and the oxidative potentials have been adjusted as appropriate in subsequent free energy diagrams. 

5. It is well known that pristine rutile is not active for hydrogen evolution reaction. Therefore, selective reduction of nitrogen to ammonia over pristine TiO2 may not be challenging even though the redox potential is near SHE.

%We agree with the reviewer, though we have been unable to find an experimental study of hydrogen evolution on pristine rutile. The fact that seems most surprising is that TiO2 is capable of dissociating the strong N-N bond more easily than it can dissociate the much weaker H-H bond. This is at the heart of the selectivity challenge, and we would argue that finding a catalyst with a high H2 evolution overpotential only solves the easy part of the challenge. We have tried to clarify this with the following addition to the introduction:

% "...indicating that TiO$_2$ is capable of dissociating the strong N-N bond more easily than the much weaker H-H bond".

6. The first step for dissociative nitrogen reduction is scission of the N-N bond. This step is not photoexcited process of TiO2. Therefore, the reviewer thinks that the high energy barrier is routinely accepted. Are there any pathway for the activation of nitrogen by photoexcited electrons? For example, N2* + * + H+ + e-  NH* + N*.

%The reviewer is correct that the chemical nature of the N-N scission suggests that a photon/electron mediated process would be needed to exploit the photocatalytic driving force, although we point out that this has not stopped other authors from hypothesizing the chemical pathway. Specifically, we chose to address the chemical pathway because it has been explicitly hypothesized by Hirakawa et. al., and because it represents one "extreme" of the reaction mechanism.

%Clearly, there is potential for "mixed" mechanisms with dissociation of N2Hx species. However, these mechanisms still lead to the formation of N* or NH* which are exceedingly unstable on the pristine and Fe-doped surfaces, and hence can be ruled out even with the modest 0.15 V overpotential. The exception is the oxygen defect, where NHx species are much more stabilized; however, N* is still very unstable, indicating that formation of HNNH is still required. Since this is the potential-limiting step for the associative mechanism we suggest that there is little advantage to showing the thermodynamics of the mixed mechanisms, though these should certainly be considered in more advanced studies with kinetics included.

%We have addressed this possibility with two comments:

% "The same stabilization would be required for NH* species, effectively eliminating any pathway involving NH* (e.g. dissociation of NNH). Adsorbed NH$_2$ species are somewhat more stable, and may exist under solvated conditions, opening the possibility of mechanisms involving dissociation of N2Hx>2 species, similar to the associative mechanism that will be discussed subsequently.

% "An alternative possibility is a mixed mechanism proceeding through dissociaton of partially hydrogenated species, since the NHx species are stable at the O-br vacancy; however, this would still necessitate the formation of the potential-limiting HNNH* species from the associative mechanism and would be thermodynamically (though not kinetically) equivalent." (page 17)

7. Equations 7–12 are incorrect.

%Again, we thank the reviewer for astuteness. The prior version listed the dissociative mechanism twice due to an error, which has now been rectified.

8. In the caption of Figure 9, "the conduction band edge" may be "the valence band edge".

%Thanks to the reviewer for catching this typo.

9. Figure 10 was not referred in the manuscript.

%Thanks for catching another oversight; we have fixed this.

10. It is suggested that nitrogen is first oxidized to nitric oxide and subsequently reduced to ammonia. If so, there are reduction pathways of NO into N2 and N2O in addition to NH3. Detailed discussions are necessary since the proposed mechanism is speculative.

11. There are typo, for example, "the the" in Line 2 on page 6.

%Thank you for catching this typo. We have proof-read the manuscript for others.

12. Abbreviation should be explained when it first appears in the text.

13. The format of the references should be unified according to the requirements of this journal.

Reviewer: 2

Comments:
I understand that the authors have chosen to remove the experimental data which was superfluous due to already published data by others as pointed out by one of the reviewers. As it stands, I do not recommend this paper for publication in EES since it only presents computational data that does not explain the experimental findings nor suggests a reaction mechanism instead it focuses on excluding a number of reaction mechanisms. Therefore, I think this manuscript would be better suited in a more technical computational journal.

% We respectfully disagree with this reviewer's assessment, and feel that the work addresses a number of specific hypotheses that appear in the experimental literature, while also suggesting others for future testing. This should make the manuscript accessible to experimentalists, and also provide experimentalists with ideas for additional studies. Contrary to the reviewer's comment, we explicitly propose a reaction mechanism (indirect reduction) that is thermodynamically plausible and does not appear elsewhere in the literature. We believe that this hypothesis-driven approach is an efficient way to communicate knowledge between experimental and computational studies. The goal of the scientific method is to test hypotheses, and we feel that conclusively disproving a hypothesis is of similar value to proving one.

% Given the nature of the work we feel that EES is an appropriate outlet. The more technical journals suggested tend to focus more heavily on method verification or development rather than hypothesis testing.

Reviewer: 3

Comments:
This study applies density functional theory calculations to examine the elementary reaction free energies associated with nitrogen reduction on the rutile TiO2 (110) surface.  The results presented illustrate well that the reduction of nitrogen to ammonia is not energetically feasible on the pristine, stoichiometric surface.  Some consideration of oxygen vacancies and doped Fe looks promising and is considered at a relatively basic level, as well as potential other conversion routes that include first oxidation followed by reduction.  The overall conclusions are interesting, and the resulted are presented with limitations clearly highlighted.  The limitations in the analysis are quite severe, as there is no direct consideration of elementary kinetics (ie, barriers), and catalytic kinetics are, of course, generally dictated by barriers and not reaction energies.  However, in this study, large differences in reaction energies are sufficient to make some interesting qualitative conclusions.  Other major limitations in the modeling approach include the lack of consideration of interfacial solvation and charging in the photoelectrocatalytic system.  Again, though these are significant limitations, the large energetic differences make qualitative conclusions still generally well supported by the more simplistic analysis. As these limitations are highlighted for the reader and discussed, I recommend publication following consideration of the following significant comments.

% We thank the reviewer for this positive assessment, and are grateful for their ability to recognize the considerable value provided by the relatively simple models employed. Further, the constructive criticism provided is very useful and we have revised and improved the manuscript accordingly.

1)       I have a number of comments regarding the consideration of the stable state of the surface under reaction conditions, the first section of results.

a.      It is unclear what the difference between the “gas” and “aqueous” conditions are in the discussion of these results.  The authors note that the gas phase is humidified, and presuming 100\% humidity, the chemical potential of gas phase water and liquid water would be equivalent. The chemical potential of water in the aqueous system is noted as being calculated directly from the saturation pressure of water (ie, 100\% RH), and from what I can tell, the water chemical potential is the only thing that could differentiate the two conditions in their modeling approach.   Did the authors consider the “gas” conditions as 0\% relative humidity, and the “aqueous” conditions as 100\% relative humidity?  If so, this should be clarified, and discussed why 0\% RH would be the relevant “gas” phase conditions given this system would presumably be exposed to ambient air.


b.      In Figure 2, to make the plots on the left (a, b, and c), I believe a constant value of the chemical potential of N2 must have been assumed.  On the right (d-f), a constant chemical potential of water must have been assumed.  The authors should clarify what values were used for these species in making these plots.

%The reviewer is correct, and we apologize for the previous ambiguity. We have added the fixed values of H2O and N2 chemical potential in the caption.

c.      In making these plots, the authors have only considered adsorbed N2, whereas the steady state system operating photocatalytically would attain a surface potential, and formation of stable NxHy species would be possible.  As the authors show, adsorbed NH3 (and other species under certain conditions) can be more stable than adsorbed N2 under operating conditions.  The authors might consider making diagrams at a presumed operating potential to consider if the surface would attain (thermodynamically) a high coverage of NHx species.  Similarly, operating conditions could cause the reduction of surface O atoms (ie, the adsorption of H or formation of surface O vacancies), and this was not considered.

2)       I am a bit confused as to the distinction between routes that first undergo oxidation and those that initiate through reduction, given the surface of TiO2 has O atoms.  It appears the difference between N2O* and N2* is that an excess O atom is present.  However, presumably N2* could form by creating an O vacancy, appearing as “N2O* adsorbed with a vacancy nearby.”  Or, wouldn’t N2 reduction occurring on a surface that had excess O* to begin with be mechanistically equivalent to the “NO” route?  Presumably N2 would adsorb to an O rather than the Ti if the surface was overoxidized.

% Regarding the first part of the comment, N2 does not adsorb at an O-br site unless it is constrained, indicating that this is not a viable route to formation of an oxygen vacancy. However, in the case of excess O* the reviewer make a good point regarding direct adsorption of N2 to a surface O*. Indeed this is (slightly) more energetically favorable than independent adsorption, and hence we have changed the mechanism for N2 -> NO to involve direct adsorption to O*. In the case of reduction from this "N2O" state, we note that the formation of O* is not favorable under reducing conditions, making this unlikely. However, a "mixed" path involving N2O formation and subsequent reduction is plausible, though this would involve N-N scission through an ONNHx species, which opens up considerably more possibilities that are beyond the scope of this work (though we are working on similar mechanisms for future work).

%This comment has been addressed by modifying the N2 -> NO free energy diagram and a discussion of the direct adsorption of N2 to O*.

3)      The authors state in the conclusions “defects are not predicted to be thermodynamically stable under operating conditions.”  The basis for this conclusion is unclear – I don’t see a consideration of the reduction of surface O atoms to form water and a vacancy considered in the paper.

4)      The consideration of Fe doping is rather basic, as the authors do not appear to have considered whether Fe doping would alter the stable state of the surface.  Fe doping forces formation of a Fe4+ formal species, though reduction to Fe3+ or Fe2+ would be much more likely under N2 reduction conditions (see Fe Pourbaix diagram).  A Fe2+ model would be relatively easy for the authors to consider, as this would likely form simply by having Fe doping occur together with an O vacancy formation.   Fe3+ could be modeled with two Fe dopants and a vacancy.  These would be much more realistic models of the Fe doped system.  Some discussion of this large limitation in the Fe doping consideration should be added, or possibly the Fe doping analysis simply removed.  In its current form, I would consider the Fe doping calculations done possibly misleading.

a.      In the Fe doped section, I did not understand the statement “The energy required to form this defect was calculated to be 2 eV relative to bulk rutile and BCC iron.”  Such a consideration would require at least knowing the chemical potential of O (or that of water, protons, and an electrochemical potential), so it is unclear the basis on which it was concluded Fe doping was unstable.

5)       Bottom of page 22, “reference hydrogen electrode” should be “reversible hydrogen electrode.”

%Thanks to the reviewer for catching this typo.

6)      Page 23 – The potentials given for holes and excited electrons should be clarified as to what is the reference scale.

% We have tried to clarify this in the methods section. We have chosen to treat holes/electron potentials in a way that absorbs DFT error into the equilibrium energy and hence preserves overpotentials that are consistent with the experimental values (otherwise overpotentials would be incorrect by an amount equal to the DFT error in the equilibrium potential). We tried to clearly state this in the captions, and have now elaborated in the methods section to clarify further.